\hypertarget{index_intro_sec}{}\section{Introduction}\label{index_intro_sec}
This manual should give an overview about the structure and functionality of the {\ttfamily compboost} {\ttfamily C++} classes. To get an insight into the underlying theory check out the {\ttfamily compboost} vignettes\+:\hypertarget{index_install_sec}{}\section{Installation}\label{index_install_sec}
Basically, the {\ttfamily C++} code can be exported and be used within any language. The only restriction is to exclude the \href{https://cran.r-project.org/web/packages/Rcpp/vignettes/Rcpp-introduction.pdf}{\tt {\ttfamily Rcpp}} specific parts which includes some {\ttfamily Rcpp\+::\+Rcout} printer and the custom classes which requires {\ttfamily Rcpp\+::\+Function} or external pointer of {\ttfamily R} as well as the \href{https://cran.r-project.org/web/packages/RcppArmadillo/vignettes/RcppArmadillo-intro.pdf}{\tt {\ttfamily Rcpp\+Armadillo}} package. To get \href{http://arma.sourceforge.net}{\tt {\ttfamily Armadillo}} run independent of {\ttfamily Rcpp} one has to link the library manually.

As it can already be suspected, the main intend is to use this package within {\ttfamily R}. This is achived by wrapping the pure {\ttfamily C++} classes by another {\ttfamily C++} wrapper which are then exported as {\ttfamily S4} class using the \href{https://cran.r-project.org/web/packages/Rcpp/vignettes/Rcpp-modules.pdf}{\tt Rcpp modules}. So the easiest way of using {\ttfamily compboost} is to install the {\ttfamily R} package\+:


\begin{DoxyCode}
devtools::install\_github(\textcolor{stringliteral}{"schalkdaniel/compboost"})
\end{DoxyCode}
\hypertarget{index_class_diagram}{}\section{\char`\"{}\+Class Diagram\char`\"{}}\label{index_class_diagram}
 